\documentclass[12pt]{article}

\usepackage[T1,T2A]{fontenc}
\usepackage[utf8]{inputenc}
\usepackage[bulgarian]{babel}
\usepackage{graphicx}
\usepackage{amssymb}
\usepackage{amsmath}
\usepackage{commath}
\usepackage{float}
\usepackage{booktabs}   
\usepackage{ltablex}

% \usepackage{sidecap} % side captions
% \usepackage[top=1.3in, bottom=1.5in, left=1.3in, right=1.3in]{geometry}

\parindent 0px % turn off indenting

% have nicer-looking links
\usepackage{hyperref}
\hypersetup{
  colorlinks   = true, %Colours links instead of ugly boxes
  urlcolor     = red, %Colour for external hyperlinks
  linkcolor    = blue, %Colour of internal links
  citecolor   = blue
}
\usepackage[hypcap=true]{caption}

% turn off sections numbers
\makeatletter
\renewcommand{\@seccntformat}[1]{}
\makeatother


\newcommand{\JMUTitle}[9]{

  \thispagestyle{empty}
  \vspace*{\stretch{1}}
  {\parindent0cm
  \rule{\linewidth}{.7ex}}
  \begin{flushright}
    \vspace*{\stretch{1}}
    \sffamily\bfseries\Huge
    #1\\
    \vspace*{\stretch{1}}
    \sffamily\bfseries\large
    #2\\
    \vspace*{\stretch{1}}
    \sffamily\bfseries\small
    #3
  \end{flushright}
  \rule{\linewidth}{.7ex}

  \vspace*{\stretch{1}}
  \begin{center}
    \includegraphics[width=3in]{./images/logo.png} \\
    \vspace*{\stretch{1}}
    \Large Курсов проект по \\ \textit{Бази от знания} \\

    \vspace*{\stretch{2}}
    \large Факултет по математика и информатика\\
    \large Софийски университет\\
    
    \vspace*{\stretch{1}}
    \large Оценен от:  #8 \\[1mm]
    
    \vspace*{\stretch{1}}
    \large #7 \\

  \end{center}
}

\newcommand*{\MyIndent}{\hspace*{7em}}



%%%%%%%%%%%%%%%%% END OF PREAMBLE %%%%%%%%%%%%%%%%



\begin{document}  

  \JMUTitle
      {Работа с онтология за пици}
      {Симеон Христов}
      {6MI3400191}
      
      {Wirtschaftswissenschaftlichen Fakultät}  % Name der Fakultaet
      {W"urzburg 2018}                          % Ort und Jahr der Erstellung
      {Юни 2023}                              % Tag der Abgabe
      {ас. Мелания Бербатова}               % Name des Erstgutachters
      {Zweitgutachter}                          % Name des Zweitgutachters

  \clearpage

\tableofcontents

\clearpage

\section{Описание на онтологията}

    \subsection{Класове}

    Онтологията съдържа йерархия от класове, описваща видове пици, тяхните основи, възможни гарнитури и степента на пикантност на тези гарнитури. Основните класове са \textit{Food}, \textit{Pizza}, \textit{PizzaTopping}, \textit{PizzaBase} и \textit{Spiciness}. Онтологията също така включва примитивни и дефинирани класове, както и обединени класове. Използва се оригиналната версия на онтологията, без промени, така, както е дефинирана \href{https://protege.stanford.edu/ontologies/pizza/pizza.owl}{тук}.

    \subsection{Свойства}

        \subsubsection{Свойства на обектите}

        Дефинираните свойства на обектите са:

        \begin{itemize}
            \item \textit{hasBase}: свързва пица с нейната основа.
            \item \textit{isBaseOf}: обратно свойство на \textit{hasBase}.
            \item \textit{hasCountryOfOrigin}: свързва обект с неговата страна на произход.
            \item \textit{hasIngredient}: свързва храна с друга храна, която е нейна съставка.
            \item \textit{isIngredientOf}: обратно свойство на \textit{hasIngredient}.
            \item \textit{hasSpiciness}: свързва гарнитура с нейното ниво на пикантност.
            \item \textit{hasTopping}: свързва пица с нейните гарнитури.
            \item \textit{isToppingOf}: обратно свойство на \textit{hasTopping}.
        \end{itemize}
        
        \subsubsection{Свойства на данните}

        В използваната онтология не са дефинирани свойства на данните.

    \subsection{Визуализация на онтологията}

    Онтологията е представена визуално на \hyperref[fig:pizza-ontology-pic]{Фигура~\ref*{fig:pizza-ontology-pic}}.

    \begin{center}
        \begin{figure}
        \centering
            \includegraphics[scale=0.36]{./images/pizza-ontology-pic.png}
            \caption{Визуализация на онтологията \textit{pizza.owl}.}
            \label{fig:pizza-ontology-pic}
        \end{figure}
    \end{center}

\section{Принципи и методи за извършване на логически извод в GraphDB}

    GraphDB съхранява експлицитни и имплицитни твърдения (твърдения, изведени от експлицитни твърдения). Когато експлицитно твърдение се премахне от базата знания, всички имплицитни твърдения, които са изведени от него също трябва да бъдат премахнати.
    
    Премахването на експлицитни твърдения се постига чрез анулиране на изведените твърдения, които вече не могат да бъдат извлечени. Един подход е да се поддържа информация за проследяване за всяко твърдение - обикновено списъкът с твърдения, които могат да бъдат изведени от това твърдение. Списъкът се изгражда по време на извода, тъй като правилата се прилагат и твърденията, изведени от правилата, се добавят към списъците на всички твърдения, които са задействали изводите. Недостатъкът на тази техника е, че информацията за веригата расте по-бързо от подразбираното затваряне. Друг подход е да се извърши обратен извод. Този вид извод не изисква информация за веригата, тъй като той по същество я преизчислява. За тази цел се използва флаг за всяко твърдение, така че алгоритъмът да може да открие кога даден ъзелв е бил посетен преди това и по този начин да избегне безкрайна рекурсия.

    Алгоритъмът, използван в GraphDB, работи по следния начин:

    \begin{enumerate}
        \item Поставя се флаг „посетен“ към всички твърдения (по подразбиране стойността е \textit{не}).
        \item Създава се списък L и в него се съхраняват твърденията, които трябва да се премахнат.
        \item Всяко твърдение в L, което не е посетено все още, се маркира като посетено и към него се прилагат правилата за прав извод.
        \item Ако няма повече непосетени твърдения в L, тогава КРАЙ.
        \item Всички изведени твърдения се добавят в списък L1.
        \item За всеки елемент в L1 се прави проверката:
            \begin{enumerate}
                \item Ако твърдението е чисто имплицитно (ако е и имплицитно, и експлицитно не се счита за чисто имплицитно), се маркира като изтрито и се проверява дали се поддържа от други твърдения. Методът isSupported() използва заявки, които съдържат предпоставките на правилата и променливите на правилата са предварително обвързани с помощта на въпросния оператор. Това означава, че методът isSupported() започва от проекцията на заявката и след това проверява дали заявката ще върне резултати (поне един), т.е. този метод извършва обратен извод.
                \item Ако резултатът е върнат от която и да е заявка (всяко правило е представено от заявка) в isSupported(), тогава това твърдение все още може да бъде извлечено от други твърдения в базата знания и затова не се изтрива.
                \item Ако всички заявки не върнат резултати, тогава това твърдение вече не може да бъде извлечено от други твърдения, така че състоянието му остава „изтрито“ и броячът на твърденията се актуализира.
            \end{enumerate}
        \item L := L1 и се преминава към стъпка 3.
    \end{enumerate}

\section{Заявки и резултати}

todo

\section{Използвани технологии}

    \begin{enumerate}
    
    \item \href{https://ontotext.com/products/graphdb/}{GraphDB} - за работа с онтологията
    \item \href{http://vowl.visualdataweb.org/webvowl-old/webvowl-old.html}{webvowl} - за визуализация на онтологията
        
    \end{enumerate}

    

%%%%%%%%%%%%%%%%% END OF MAIN TEXT %%%%%%%%%%%%%%%%



\listoffigures

\section{Източници}

\begin{quote}

    \begin{enumerate}
    
    \item \href{https://graphdb.ontotext.com/documentation/10.2/reasoning.html}{Reasoning in GraphDB}

    \item \href{https://en.wikibooks.org/wiki/SPARQL/Expressions_and_Functions}{SPARQL/Expressions and Functions}

    \item \href{https://en.wikibooks.org/wiki/SPARQL/Aggregate_functions}{SPARQL/Aggregate functions}

    \item \href{https://en.wikibooks.org/wiki/SPARQL/FILTER}{SPARQL/FILTER}
        
    \end{enumerate}

\end{quote}

\end{document}