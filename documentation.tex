\documentclass[12pt]{article}

\usepackage[T1,T2A]{fontenc}
\usepackage[utf8]{inputenc}
\usepackage[bulgarian]{babel}
\usepackage{graphicx}
\usepackage{amssymb}
\usepackage{amsmath}
\usepackage{commath}
\usepackage{float}
\usepackage{booktabs}   
\usepackage{ltablex}

% For SPARQL syntax
\usepackage{listings}
\lstdefinestyle{sparql}{
  basicstyle=\ttfamily,
  columns=fullflexible,
  keepspaces=true,
  keywordstyle=\bfseries,
  morekeywords={PREFIX,SELECT,FROM,WHERE,ORDER,BY,FILTER, GROUP, AS, BIND, IF, DESC, ASC},
  sensitive=false,
  showstringspaces=false,
  tabsize=4,
  frame=single
}


% \usepackage{sidecap} % side captions
% \usepackage[top=1.3in, bottom=1.5in, left=1.3in, right=1.3in]{geometry}

% have nicer-looking links
\usepackage{hyperref}
\hypersetup{
  colorlinks   = true, %Colours links instead of ugly boxes
  urlcolor     = red, %Colour for external hyperlinks
  linkcolor    = blue, %Colour of internal links
  citecolor   = blue
}
\usepackage[hypcap=true]{caption}


\newcommand{\JMUTitle}[9]{

  \thispagestyle{empty}
  \vspace*{\stretch{1}}
  {\parindent0cm
  \rule{\linewidth}{.7ex}}
  \begin{flushright}
    \vspace*{\stretch{1}}
    \sffamily\bfseries\Huge
    #1\\
    \vspace*{\stretch{1}}
    \sffamily\bfseries\large
    #2\\
    \vspace*{\stretch{1}}
    \sffamily\bfseries\small
    #3
  \end{flushright}
  \rule{\linewidth}{.7ex}

  \vspace*{\stretch{1}}
  \begin{center}
    \includegraphics[width=3in]{./images/logo.png} \\
    \vspace*{\stretch{1}}
    \Large Курсов проект по \\ \textit{Бази от знания} \\

    \vspace*{\stretch{2}}
    \large Факултет по математика и информатика\\
    \large Софийски университет\\
    
    \vspace*{\stretch{1}}
    \large Оценен от:  #8 \\[1mm]
    
    \vspace*{\stretch{1}}
    \large #7 \\

  \end{center}
}

\newcommand*{\MyIndent}{\hspace*{7em}}



%%%%%%%%%%%%%%%%% END OF PREAMBLE %%%%%%%%%%%%%%%%



\begin{document}  

  \JMUTitle
      {Работа с онтология за пици}
      {Симеон Христов}
      {6MI3400191}
      
      {Wirtschaftswissenschaftlichen Fakultät}  % Name der Fakultaet
      {W"urzburg 2018}                          % Ort und Jahr der Erstellung
      {Юни 2023}                              % Tag der Abgabe
      {ас. Мелания Бербатова}               % Name des Erstgutachters
      {Zweitgutachter}                          % Name des Zweitgutachters

  \clearpage

\tableofcontents

\clearpage

\section{Описание на онтологията}

    \subsection{Класове}

    Онтологията съдържа йерархия от класове, описваща видове пици, тяхните основи, възможни гарнитури и степента на пикантност на тези гарнитури. Основните класове са \textit{Food}, \textit{Pizza}, \textit{PizzaTopping}, \textit{PizzaBase} и \textit{Spiciness}. Онтологията също така включва примитивни и дефинирани класове, както и обединени класове. Използва се оригиналната версия на онтологията, без промени, така, както е дефинирана \href{https://protege.stanford.edu/ontologies/pizza/pizza.owl}{тук}.

    \subsection{Свойства}

        \subsubsection{Свойства на обектите}

        Дефинираните свойства на обектите са:

        \begin{itemize}
            \item \textit{hasBase}: свързва пица с нейната основа.
            \item \textit{isBaseOf}: обратно свойство на \textit{hasBase}.
            \item \textit{hasCountryOfOrigin}: свързва обект с неговата страна на произход.
            \item \textit{hasIngredient}: свързва храна с друга храна, която е нейна съставка.
            \item \textit{isIngredientOf}: обратно свойство на \textit{hasIngredient}.
            \item \textit{hasSpiciness}: свързва гарнитура с нейното ниво на пикантност.
            \item \textit{hasTopping}: свързва пица с нейните гарнитури.
            \item \textit{isToppingOf}: обратно свойство на \textit{hasTopping}.
        \end{itemize}
        
        \subsubsection{Свойства на данните}

        В използваната онтология не са дефинирани свойства на данните.

    \subsection{Визуализация на онтологията}

    Онтологията е представена визуално на \hyperref[fig:pizza-ontology-pic]{Фигура~\ref*{fig:pizza-ontology-pic}}.

    \begin{center}
        \begin{figure}
        \centering
            \includegraphics[scale=0.36]{./images/pizza-ontology-pic.png}
            \caption{Визуализация на онтологията \textit{pizza.owl}.}
            \label{fig:pizza-ontology-pic}
        \end{figure}
    \end{center}

\section{Принципи и методи за извършване на логически извод в GraphDB}

    GraphDB съхранява експлицитни и имплицитни твърдения (твърдения, изведени от експлицитни твърдения). Когато експлицитно твърдение се премахне от базата знания, всички имплицитни твърдения, които са изведени от него също трябва да бъдат премахнати.
    
    Премахването на експлицитни твърдения се постига чрез анулиране на изведените твърдения, които вече не могат да бъдат извлечени. Един подход е да се поддържа информация за проследяване за всяко твърдение - обикновено списъкът с твърдения, които могат да бъдат изведени от това твърдение. Списъкът се изгражда по време на извода, тъй като правилата се прилагат и твърденията, изведени от правилата, се добавят към списъците на всички твърдения, които са задействали изводите. Недостатъкът на тази техника е, че информацията за веригата расте по-бързо от подразбираното затваряне. Друг подход е да се извърши обратен извод. Този вид извод не изисква информация за веригата, тъй като той по същество я преизчислява. За тази цел се използва флаг за всяко твърдение, така че алгоритъмът да може да открие кога даден възел е бил посетен преди това и по този начин да избегне безкрайна рекурсия.

    Алгоритъмът, използван в GraphDB, работи по следния начин:

    \begin{enumerate}
        \item Поставя се флаг „посетен“ към всички твърдения (по подразбиране стойността е \textit{не}).
        \item Създава се списък L и в него се съхраняват твърденията, които трябва да се премахнат.
        \item Всяко твърдение в L, което не е посетено все още, се маркира като посетено и към него се прилагат правилата за прав извод.
        \item Ако няма повече непосетени твърдения в L, тогава КРАЙ.
        \item Всички изведени твърдения се добавят в списък L1.
        \item За всеки елемент в L1 се прави проверката:
            \begin{enumerate}
                \item Ако твърдението е чисто имплицитно (ако е и имплицитно, и експлицитно не се счита за чисто имплицитно), се маркира като изтрито и се проверява дали се поддържа от други твърдения. Методът isSupported() използва заявки, които съдържат предпоставките на правилата и променливите на правилата са предварително обвързани с помощта на въпросния оператор. Това означава, че методът isSupported() започва от проекцията на заявката и след това проверява дали заявката ще върне резултати (поне един), т.е. този метод извършва обратен извод.
                \item Ако резултатът е върнат от която и да е заявка (всяко правило е представено от заявка) в isSupported(), тогава това твърдение все още може да бъде извлечено от други твърдения в базата знания и затова не се изтрива.
                \item Ако всички заявки не върнат резултати, тогава това твърдение вече не може да бъде извлечено от други твърдения, така че състоянието му остава „изтрито“ и броячът на твърденията се актуализира.
            \end{enumerate}
        \item L := L1 и се преминава към стъпка 3.
    \end{enumerate}

\section{Заявки и резултати}

    \subsection{Заявки върху класове}

        Заявка 1, намираща всички гарнитури, които са \textit{FishTopping}.
        
        \begin{lstlisting}[language=SPARQL,style=sparql]
PREFIX pizza:<http://www.co-ode.org/ontologies/pizza/pizza.owl#>
PREFIX rdfs: <http://www.w3.org/2000/01/rdf-schema#>
SELECT ?topping
WHERE { ?topping rdfs:subClassOf pizza:FishTopping }\end{lstlisting}

        Резултатът от изпълнението е показан на \hyperref[fig:q01]{Фигура~\ref*{fig:q01}}.

        \begin{center}
            \begin{figure}
            \centering
                \includegraphics[scale=0.3]{./images/q01.png}
                \caption{Резултат от изпълнение на заявка 1.}
                \label{fig:q01}
            \end{figure}
        \end{center}

\clearpage
\pagebreak

        Заявка 2, намираща всички пици, които имат гарнитура \textit{AnchoviesTopping}.

        \begin{lstlisting}[language=SPARQL,style=sparql]
PREFIX pizza:<http://www.co-ode.org/ontologies/pizza/pizza.owl#>
PREFIX rdfs: <http://www.w3.org/2000/01/rdf-schema#>
PREFIX owl: <http://www.w3.org/2002/07/owl#>
SELECT DISTINCT ?pizza
WHERE {
    ?pizza rdfs:subClassOf pizza:NamedPizza .
    ?pizza rdfs:subClassOf ?restriction .
    ?restriction owl:onProperty pizza:hasTopping .
    ?restriction owl:someValuesFrom pizza:AnchoviesTopping .
}\end{lstlisting}
    
        Резултатът от изпълнението е показан на \hyperref[fig:q02]{Фигура~\ref*{fig:q02}}.

        \begin{center}
            \begin{figure}
            \centering
                \includegraphics[scale=0.3]{./images/q02.png}
                \caption{Резултат от изпълнение на заявка 2.}
                \label{fig:q02}
            \end{figure}
        \end{center}
   
\clearpage
\pagebreak
     
    \subsection{Заявки върху екземпляри}

        Заявка 3, намираща всички екземпляри.
        
        \begin{lstlisting}[language=SPARQL,style=sparql]
PREFIX pizza:<http://www.co-ode.org/ontologies/pizza/pizza.owl#>
PREFIX rdf: <http://www.w3.org/1999/02/22-rdf-syntax-ns#>
PREFIX owl: <http://www.w3.org/2002/07/owl#>
SELECT ?name
WHERE { ?name a owl:NamedIndividual . }\end{lstlisting}

        Резултатът от изпълнението е показан на \hyperref[fig:q03]{Фигура~\ref*{fig:q03}}.

        \begin{center}
            \begin{figure}
            \centering
                \includegraphics[scale=0.3]{./images/q03.png}
                \caption{Резултат от изпълнение на заявка 3.}
                \label{fig:q03}
            \end{figure}
        \end{center}

\clearpage
\pagebreak

        Заявка 4, намираща всички екземпляри, чиито имена започват с гласна буква ('a', 'e', 'i', 'o' или 'u').
        
        \begin{lstlisting}[language=SPARQL,style=sparql]
PREFIX pizza:<http://www.co-ode.org/ontologies/pizza/pizza.owl#>
PREFIX rdf: <http://www.w3.org/1999/02/22-rdf-syntax-ns#>
PREFIX owl: <http://www.w3.org/2002/07/owl#>
SELECT ?name ?string ?substring
WHERE {
    ?name a owl:NamedIndividual .
    BIND(STR(?name) AS ?string) .
    BIND(SUBSTR(?string, 50) AS ?substring) .
    FILTER REGEX (?substring, "^(a|e|i|o|u)", "i") .
}\end{lstlisting}

        Резултатът от изпълнението е показан на \hyperref[fig:q04]{Фигура~\ref*{fig:q04}}.

        \begin{center}
            \begin{figure}
            \centering
                \includegraphics[scale=0.3]{./images/q04.png}
                \caption{Резултат от изпълнение на заявка 4.}
                \label{fig:q04}
            \end{figure}
        \end{center}

\clearpage
\pagebreak

        Заявка 5, намираща броя пици за всяка държава, за която са създадени пици.
        
        \begin{lstlisting}[language=SPARQL,style=sparql]
PREFIX pizza:<http://www.co-ode.org/ontologies/pizza/pizza.owl#>
PREFIX rdf: <http://www.w3.org/1999/02/22-rdf-syntax-ns#>
PREFIX owl: <http://www.w3.org/2002/07/owl#>
PREFIX rdfs: <http://www.w3.org/2000/01/rdf-schema#>
SELECT ?countryName (COUNT (DISTINCT ?name) AS ?numPizzas)
WHERE {
    ?name rdfs:subClassOf pizza:NamedPizza, ?restr .
    ?restr owl:onProperty pizza:hasCountryOfOrigin ;
        owl:hasValue ?countryName .
}
GROUP BY ?countryName\end{lstlisting}

        Резултатът от изпълнението е показан на \hyperref[fig:q05]{Фигура~\ref*{fig:q05}}.

        \begin{center}
            \begin{figure}
            \centering
                \includegraphics[scale=0.3]{./images/q05.png}
                \caption{Резултат от изпълнение на заявка 5.}
                \label{fig:q05}
            \end{figure}
        \end{center}
    
\clearpage
\pagebreak

        Заявка 6, намираща степента на пикантност на пиците за всяка държава.
        
        \begin{lstlisting}[language=SPARQL,style=sparql]
PREFIX pizza:<http://www.co-ode.org/ontologies/pizza/pizza.owl#>
PREFIX rdf: <http://www.w3.org/1999/02/22-rdf-syntax-ns#>
PREFIX owl: <http://www.w3.org/2002/07/owl#>
PREFIX rdfs: <http://www.w3.org/2000/01/rdf-schema#>
SELECT DISTINCT ?countryName ?pizzaName ?spiciness
WHERE {
    ?pizzaName rdfs:subClassOf pizza:NamedPizza,
                            ?restrCountry,
                            ?restrTopping .
    ?restrTopping owl:onProperty     pizza:hasTopping ;
                  owl:someValuesFrom ?topping .
    ?topping rdfs:subClassOf ?restrToppingSpiciness.
    ?restrToppingSpiciness owl:onProperty pizza:hasSpiciness ;
                           owl:someValuesFrom ?spiciness .
    ?restrCountry owl:onProperty pizza:hasCountryOfOrigin ;
                owl:hasValue ?countryName .
}\end{lstlisting}

        Резултатът от изпълнението е показан на \hyperref[fig:q06]{Фигура~\ref*{fig:q06}}.

        \begin{center}
            \begin{figure}
            \centering
                \includegraphics[scale=0.3]{./images/q06.png}
                \caption{Резултат от изпълнение на заявка 6.}
                \label{fig:q06}
            \end{figure}
        \end{center}
    
\clearpage
\pagebreak

        Заявка 7, намираща държавите, за които са създадени пици със степен на пикантност \textit{Hot} и броя на тези пици.
        
        \begin{lstlisting}[language=SPARQL,style=sparql]
PREFIX pizza:<http://www.co-ode.org/ontologies/pizza/pizza.owl#>
PREFIX rdf: <http://www.w3.org/1999/02/22-rdf-syntax-ns#>
PREFIX owl: <http://www.w3.org/2002/07/owl#>
PREFIX rdfs: <http://www.w3.org/2000/01/rdf-schema#>
SELECT
    DISTINCT ?countryName
    (COUNT(DISTINCT ?pizzaName) AS ?hotPizzaCount)
WHERE {
    ?pizzaName rdfs:subClassOf pizza:NamedPizza,
                               ?restrCountry,
                               ?restrTopping .
    ?restrTopping owl:onProperty pizza:hasTopping ;
                  owl:someValuesFrom ?topping .
    ?topping rdfs:subClassOf ?restrToppingSpiciness .
    ?restrToppingSpiciness owl:onProperty pizza:hasSpiciness ;
                           owl:someValuesFrom pizza:Hot .
    ?restrCountry owl:onProperty pizza:hasCountryOfOrigin ;
                  owl:hasValue ?countryName .
}
GROUP BY ?countryName\end{lstlisting}

        Резултатът от изпълнението е показан на \hyperref[fig:q07]{Фигура~\ref*{fig:q07}}.

        \begin{center}
            \begin{figure}
            \centering
                \includegraphics[scale=0.3]{./images/q07.png}
                \caption{Резултат от изпълнение на заявка 7.}
                \label{fig:q07}
            \end{figure}
        \end{center}
                
\clearpage
\pagebreak

    \subsection{Заявки, с повече от един концепт и повече от един FILTER}

        Заявка 8, намираща всички гарнитури на пиците, чиито имена са с повече от 5 символа.
        
        \begin{lstlisting}[language=SPARQL,style=sparql]
PREFIX pizza:<http://www.co-ode.org/ontologies/pizza/pizza.owl#>
PREFIX rdfs: <http://www.w3.org/2000/01/rdf-schema#>
PREFIX owl: <http://www.w3.org/2002/07/owl#>
SELECT DISTINCT ?pizza ?topping
WHERE {
    ?pizza rdfs:subClassOf pizza:NamedPizza, ?restr .
    ?restr owl:onProperty pizza:hasTopping ;
            owl:someValuesFrom ?topping .
    FILTER NOT EXISTS {
      ?topping rdfs:subClassOf ?restrToppingSpiciness .
      ?restrToppingSpiciness owl:onProperty pizza:hasSpiciness;
                             owl:someValuesFrom pizza:Hot .
    }
    FILTER ((STRLEN(str(?pizza)) - 49) > 5)
}
ORDER BY ?pizza\end{lstlisting}

        Резултатът от изпълнението е обемен (93 реда). Затова към предадените файлове има и \textit{csv} файлове. Резултатът от тази заявка се намира в \textit{query\-result\_q8.csv}. Първите 5 реда са показани на \hyperref[fig:q08]{Фигура~\ref*{fig:q08}}.

        \begin{center}
            \begin{figure}
            \centering
                \includegraphics[scale=0.3]{./images/q08.png}
                \caption{Частичен резултат от изпълнение на заявка 8.}
                \label{fig:q08}
            \end{figure}
        \end{center}
            
\clearpage
\pagebreak

        Заявка 9, намираща пиците и гарнитурите им за всички пици, които имат вид месна гарнитура, чието име започва със съгласна буква, и нямат гарнитура от вид \textit{HotSpicedBeefTopping}.
        
        \begin{lstlisting}[language=SPARQL,style=sparql]
PREFIX pizza:<http://www.co-ode.org/ontologies/pizza/pizza.owl#>
PREFIX rdfs: <http://www.w3.org/2000/01/rdf-schema#>
PREFIX owl: <http://www.w3.org/2002/07/owl#>
SELECT DISTINCT ?pizza ?name
WHERE {
    ?name rdfs:subClassOf pizza:MeatTopping.
    BIND(STR(?name) AS ?string)
    BIND(SUBSTR(?string, 50 ) AS ?substring)
    FILTER(REGEX(?substring, "^[^(a|e|i|o|u)]", "i"))
        
    ?pizza rdfs:subClassOf pizza:NamedPizza, ?restrTopping.
    ?restrTopping owl:onProperty pizza:hasTopping;
                    owl:someValuesFrom ?name.
    FILTER NOT EXISTS {
        ?pizza rdfs:subClassOf pizza:NamedPizza, ?restrTopping.
        ?restrTopping owl:onProperty pizza:hasTopping;
                owl:someValuesFrom pizza:HotSpicedBeefTopping.
    }
}\end{lstlisting}

        Резултатът от изпълнението е показан на \hyperref[fig:q09]{Фигура~\ref*{fig:q09}}.

        \begin{center}
            \begin{figure}
            \centering
                \includegraphics[scale=0.3]{./images/q09.png}
                \caption{Резултат от изпълнение на заявка 9.}
                \label{fig:q09}
            \end{figure}
        \end{center}
    
\clearpage
\pagebreak

        Заявка 10, намираща имената на всички пици, които не са от Америка и нямат гарнитура \textit{TomatoTopping} и гарнитура \textit{MushroomTopping}.
        
        \begin{lstlisting}[language=SPARQL,style=sparql]
PREFIX pizza:<http://www.co-ode.org/ontologies/pizza/pizza.owl#>
PREFIX rdfs: <http://www.w3.org/2000/01/rdf-schema#>
PREFIX owl: <http://www.w3.org/2002/07/owl#>
SELECT ?pizzaName ?toppingName
WHERE {
    ?pizzaName rdfs:subClassOf pizza:NamedPizza, ?restrTopping .
    ?restrTopping owl:onProperty pizza:hasTopping ;
                owl:someValuesFrom ?toppingName .
    
    FILTER NOT EXISTS {
        ?restrTopping owl:someValuesFrom pizza:TomatoTopping .
    }
    FILTER NOT EXISTS {
        ?restrTopping owl:someValuesFrom pizza:MushroomTopping .
    }
    FILTER NOT EXISTS {
        ?pizzaName rdfs:subClassOf ?restrCountry .
        ?restrCountry owl:onProperty pizza:hasCountryOfOrigin ;
                    owl:hasValue pizza:America .
    }
}\end{lstlisting}

        Резултатът от изпълнението е обемен (160 реда) и се намира в \textit{query\-result\_q10.csv}. Първите 5 реда са показани на \hyperref[fig:q10]{Фигура~\ref*{fig:q10}}.

        \begin{center}
            \begin{figure}
            \centering
                \includegraphics[scale=0.3]{./images/q10.png}
                \caption{Частичен резултат от изпълнение на заявка 10.}
                \label{fig:q10}
            \end{figure}
        \end{center}
        
\clearpage
\pagebreak

    \subsection{Заявки, демонстриращи логически извод}

        Заявка 11, намираща имената на всички видове пици.
        
        \begin{lstlisting}[language=SPARQL,style=sparql]
PREFIX pizza:<http://www.co-ode.org/ontologies/pizza/pizza.owl#>
PREFIX rdf: <http://www.w3.org/1999/02/22-rdf-syntax-ns#>
PREFIX owl: <http://www.w3.org/2002/07/owl#>
PREFIX rdfs: <http://www.w3.org/2000/01/rdf-schema#>
SELECT DISTINCT ?pizzaName
WHERE { ?pizzaName rdfs:subClassOf pizza:Pizza. }\end{lstlisting}

        Резултатът от изпълнението без логически извод е показан на \hyperref[fig:q11-01]{Фигура~\ref*{fig:q11-01}}.
        Резултатът от изпълнението с логически извод е показан на \hyperref[fig:q11-02]{Фигура~\ref*{fig:q11-02}}.

        \begin{center}
            \begin{figure}
            \centering
                \includegraphics[scale=0.3]{./images/q11-01.png}
                \caption{Резултат от изпълнение на заявка 11 без логически извод.}
                \label{fig:q11-01}
            \end{figure}
        \end{center}

        \begin{center}
            \begin{figure}
            \centering
                \includegraphics[scale=0.3]{./images/q11-02.png}
                \caption{Резултат от изпълнение на заявка 11 с логически извод.}
                \label{fig:q11-02}
            \end{figure}
        \end{center}


\clearpage
\pagebreak

        Заявка 12, намираща пиците, създадени в Италия.
        
        \begin{lstlisting}[language=SPARQL,style=sparql]
PREFIX pizza:<http://www.co-ode.org/ontologies/pizza/pizza.owl#>
PREFIX rdf: <http://www.w3.org/1999/02/22-rdf-syntax-ns#>
PREFIX owl: <http://www.w3.org/2002/07/owl#>
PREFIX rdfs: <http://www.w3.org/2000/01/rdf-schema#>
SELECT DISTINCT ?pizzaName
WHERE {
    ?pizzaName rdfs:subClassOf pizza:Pizza,
                               ?restrCountry .
    ?restrCountry owl:onProperty pizza:hasCountryOfOrigin ;
                  owl:hasValue pizza:Italy .
}\end{lstlisting}

        Резултатът от изпълнението без логически извод е показан на \hyperref[fig:q12-01]{Фигура~\ref*{fig:q12-01}}.
        Резултатът от изпълнението с логически извод е показан на \hyperref[fig:q12-02]{Фигура~\ref*{fig:q12-02}}.

        \begin{center}
            \begin{figure}
            \centering
                \includegraphics[scale=0.3]{./images/q12-01.png}
                \caption{Резултат от изпълнение на заявка 12 без логически извод.}
                \label{fig:q12-01}
            \end{figure}
        \end{center}

        \begin{center}
            \begin{figure}
            \centering
                \includegraphics[scale=0.3]{./images/q12-02.png}
                \caption{Резултат от изпълнение на заявка 12 с логически извод.}
                \label{fig:q12-02}
            \end{figure}
        \end{center}

\clearpage
\pagebreak

        Заявка 13, намираща имената на пиците и гарнитурите, които имат име на гарнитура, започващо със съгласна буква и завършващо с гласна.
        
        \begin{lstlisting}[language=SPARQL,style=sparql]
PREFIX pizza:<http://www.co-ode.org/ontologies/pizza/pizza.owl#>
PREFIX rdf: <http://www.w3.org/1999/02/22-rdf-syntax-ns#>
PREFIX owl: <http://www.w3.org/2002/07/owl#>
PREFIX rdfs: <http://www.w3.org/2000/01/rdf-schema#>
SELECT DISTINCT ?pizza ?name
WHERE {
    ?name rdfs:subClassOf pizza:PizzaTopping.
    BIND(STR(?name) as ?string) .
    BIND(SUBSTR(?string, 50) as ?substring) .
    FILTER REGEX (?substring, "^[^(a|e|i|o|u)]", "i") .
    FILTER REGEX (?substring, "(a|e|i|o|u)$", "i") .
    
    ?pizza rdfs:subClassOf pizza:NamedPizza,
                           ?restrTopping .
    ?restrTopping owl:onProperty pizza:hasTopping ;
                  owl:someValuesFrom ?name .
}\end{lstlisting}

        Резултатът от изпълнението без логически извод е показан на \hyperref[fig:q13-01]{Фигура~\ref*{fig:q13-01}}.
        Резултатът от изпълнението с логически извод е показан на \hyperref[fig:q13-02]{Фигура~\ref*{fig:q13-02}}.

        \begin{center}
            \begin{figure}
            \centering
                \includegraphics[scale=0.3]{./images/q13-01.png}
                \caption{Резултат от изпълнение на заявка 13 без логически извод.}
                \label{fig:q13-01}
            \end{figure}
        \end{center}

        \begin{center}
            \begin{figure}
            \centering
                \includegraphics[scale=0.3]{./images/q13-02.png}
                \caption{Резултат от изпълнение на заявка 13 с логически извод.}
                \label{fig:q13-02}
            \end{figure}
        \end{center}

        
\clearpage
\pagebreak

    \subsection{Заявки, с операции върху множества}

        Заявка 14, която намира всички пици, които имат имена с начална буква "S" или имат гарнитура \textit{PeperonataTopping}. За да се постигнат докладваните резултати е необходимо да се използва логически извод.

        \begin{lstlisting}[language=SPARQL,style=sparql]
PREFIX pizza:<http://www.co-ode.org/ontologies/pizza/pizza.owl#>
PREFIX rdfs: <http://www.w3.org/2000/01/rdf-schema#>
PREFIX owl: <http://www.w3.org/2002/07/owl#>
SELECT DISTINCT ?pizzaName
WHERE {
    {
        ?pizzaName rdfs:subClassOf pizza:Pizza ;
                rdfs:label ?label .
        FILTER regex(?label, "^S")
    }
    UNION
    {
        ?pizzaName rdfs:subClassOf pizza:Pizza,
                                   ?restrictionTopping .
        ?restrictionTopping owl:onProperty pizza:hasTopping ;
                owl:someValuesFrom pizza:PeperonataTopping .
    }
}\end{lstlisting}

        Резултатът от изпълнението е показан на \hyperref[fig:q14]{Фигура~\ref*{fig:q14}}.

        \begin{center}
            \begin{figure}
            \centering
                \includegraphics[scale=0.3]{./images/q14.png}
                \caption{Резултат от изпълнение на заявка 14.}
                \label{fig:q14}
            \end{figure}
        \end{center}
                
\clearpage
\pagebreak

        Заявка 15, която намира всички класове и техните етикети. За да се постигнат докладваните резултати е необходимо да се използва логически извод.

        \begin{lstlisting}[language=SPARQL,style=sparql]
PREFIX pizza:<http://www.co-ode.org/ontologies/pizza/pizza.owl#>
PREFIX rdf: <http://www.w3.org/1999/02/22-rdf-syntax-ns#>
PREFIX owl: <http://www.w3.org/2002/07/owl#>
PREFIX rdfs: <http://www.w3.org/2000/01/rdf-schema#>
SELECT DISTINCT ?name ?label
WHERE {
    ?name rdfs:subClassOf ?cls .
    OPTIONAL {
        ?name rdfs:label ?label .
    }
}\end{lstlisting}

        Резултатът от изпълнението е обемен (228 реда) и се намира в \textit{query\-result\_q15.csv}. Първите 5 реда са показани на \hyperref[fig:q15]{Фигура~\ref*{fig:q15}}.

        \begin{center}
            \begin{figure}
            \centering
                \includegraphics[scale=0.3]{./images/q15.png}
                \caption{Частичен резултат от изпълнение на заявка 15.}
                \label{fig:q15}
            \end{figure}
        \end{center}
                
\clearpage
\pagebreak

    \subsection{Заявки с функции за запазване на стойност в параметър}
        
        Заявка 16, която намира имената и дължините на имената на всички пици с име.

        \begin{lstlisting}[language=SPARQL,style=sparql]
PREFIX pizza:<http://www.co-ode.org/ontologies/pizza/pizza.owl#>
PREFIX rdfs: <http://www.w3.org/2000/01/rdf-schema#>
SELECT DISTINCT ?pizza ?nameLength
WHERE {
    ?pizza rdfs:subClassOf pizza:NamedPizza;
        rdfs:label ?name.
    BIND(STRLEN(SUBSTR(STR(?pizza), 50)) AS ?nameLength)
}
ORDER BY ?nameLength ?pizza\end{lstlisting}

        Резултатът от изпълнението е показан на \hyperref[fig:q16]{Фигура~\ref*{fig:q16}}.

        \begin{center}
            \begin{figure}
            \centering
                \includegraphics[scale=0.3]{./images/q16.png}
                \caption{Резултат от изпълнение на заявка 16.}
                \label{fig:q16}
            \end{figure}
        \end{center}
                
\clearpage
\pagebreak

        Заявка 17, която за пиците \textit{Margherita}, \textit{LaReine},  \textit{VegetarianPizza} намира
 тяхната цена или показва текста \textit{Price not available}, ако такава не е налична.
 
        \begin{lstlisting}[language=SPARQL,style=sparql]
PREFIX pizza:<http://www.co-ode.org/ontologies/pizza/pizza.owl#>
PREFIX xsd: <http://www.w3.org/2001/XMLSchema#>

SELECT ?pizza ?priceInDollars
WHERE {
    VALUES ?pizza {
        pizza:Margherita
        pizza:LaReine
        pizza:VegetarianPizza
    }
    OPTIONAL { ?pizza pizza:hasPrice ?price. }
    BIND(
        IF(BOUND(?price),
        ?price * "1.2"^^xsd:decimal,
        "Price not available"
    ) AS ?priceInDollars)
}\end{lstlisting}

        Резултатът от изпълнението е показан на \hyperref[fig:q17]{Фигура~\ref*{fig:q17}}.

        \begin{center}
            \begin{figure}
            \centering
                \includegraphics[scale=0.3]{./images/q17.png}
                \caption{Резултат от изпълнение на заявка 17.}
                \label{fig:q17}
            \end{figure}
        \end{center}
            
\clearpage
\pagebreak

    \subsection{Заявки, с агрегиращи функции}
        
        Заявка 18, която за всички гарнитури намира в колко пици се срещат те.

        \begin{lstlisting}[language=SPARQL,style=sparql]
PREFIX pizza:<http://www.co-ode.org/ontologies/pizza/pizza.owl#>
PREFIX rdfs: <http://www.w3.org/2000/01/rdf-schema#>
PREFIX owl: <http://www.w3.org/2002/07/owl#>

SELECT ?topping (COUNT(DISTINCT ?pizza) AS ?numPizzas)
WHERE {
    ?pizza rdfs:subClassOf pizza:Pizza, ?restrTopping .
    ?restrTopping owl:onProperty pizza:hasTopping ;
                owl:someValuesFrom ?topping .
}
GROUP BY ?topping
ORDER BY DESC(?numPizzas) ASC(?topping)\end{lstlisting}

        Резултатът от изпълнението е обемен (37 реда) и се намира в \textit{query\-result\_q18.csv}. Първите 5 реда са показани на \hyperref[fig:q18]{Фигура~\ref*{fig:q18}}.

        \begin{center}
            \begin{figure}
            \centering
                \includegraphics[scale=0.3]{./images/q18.png}
                \caption{Резултат от изпълнение на заявка 18.}
                \label{fig:q18}
            \end{figure}
        \end{center}
        
\clearpage
\pagebreak

        Заявка 18, която за всички пици намира колко гарнитури имат те.

        \begin{lstlisting}[language=SPARQL,style=sparql]
PREFIX pizza:<http://www.co-ode.org/ontologies/pizza/pizza.owl#>
PREFIX rdfs: <http://www.w3.org/2000/01/rdf-schema#>
PREFIX owl: <http://www.w3.org/2002/07/owl#>

SELECT ?pizza (COUNT(DISTINCT ?topping) AS ?numTopping)
WHERE {
    ?pizza rdfs:subClassOf pizza:NamedPizza, ?restrTopping .
    ?restrTopping owl:onProperty pizza:hasTopping ;
                owl:someValuesFrom ?topping .
}
GROUP BY ?pizza
ORDER BY DESC(?numTopping) ASC(?pizza)\end{lstlisting}

        Резултатът от изпълнението е показан на \hyperref[fig:q19]{Фигура~\ref*{fig:q19}}.

        \begin{center}
            \begin{figure}
            \centering
                \includegraphics[scale=0.3]{./images/q19.png}
                \caption{Резултат от изпълнение на заявка 19.}
                \label{fig:q19}
            \end{figure}
        \end{center}
        
\clearpage
\pagebreak

\section{Използвани технологии}

    \begin{enumerate}
    
    \item \href{https://ontotext.com/products/graphdb/}{GraphDB} - за работа с онтологията
    \item \href{http://vowl.visualdataweb.org/webvowl-old/webvowl-old.html}{webvowl} - за визуализация на онтологията
        
    \end{enumerate}

    

%%%%%%%%%%%%%%%%% END OF MAIN TEXT %%%%%%%%%%%%%%%%



\listoffigures

\section{Източници}

\begin{quote}

    \begin{enumerate}
    
    \item \href{https://graphdb.ontotext.com/documentation/10.2/reasoning.html}{Reasoning in GraphDB}

    \item \href{https://en.wikibooks.org/wiki/SPARQL/Expressions_and_Functions}{SPARQL/Expressions and Functions}

    \item \href{https://en.wikibooks.org/wiki/SPARQL/Aggregate_functions}{SPARQL/Aggregate functions}

    \item \href{https://en.wikibooks.org/wiki/SPARQL/FILTER}{SPARQL/FILTER}
        
    \end{enumerate}

\end{quote}

\end{document}